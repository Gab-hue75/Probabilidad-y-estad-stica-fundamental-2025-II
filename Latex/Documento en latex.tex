\documentclass[12pt]{article}
\usepackage[utf8]{inputenc}
\usepackage{lipsum} % Para texto de ejemplo

\title{Título del Artículo Científico}
\author{Autor Nombre}
\date{\today}

\begin{document}

\maketitle

\begin{abstract}
En este artículo se presenta un estudio sobre [tema del artículo]. Se analizan los principales aspectos relacionados con [tema], utilizando métodos estadísticos y probabilísticos fundamentales. Los resultados obtenidos contribuyen al entendimiento y desarrollo de nuevas aplicaciones en el área.
\end{abstract}

\section{Introducción}
La investigación científica en el campo de [campo de estudio] ha experimentado un crecimiento significativo en los últimos años. El estudio de [tema específico] es fundamental para comprender [problema o fenómeno relevante]. Diversos autores han abordado este tema desde distintas perspectivas, resaltando la importancia de [aspectos clave].

En este trabajo, se propone [objetivo principal del artículo], utilizando [metodología empleada]. Los resultados obtenidos permiten [conclusión preliminar o relevancia del estudio].

% Puedes reemplazar el texto de ejemplo con tu propia introducción.
%\lipsum[1-2]

\end{document}